\documentclass{article}
\usepackage{hyperref}
\usepackage{float}

% Language setting
% Replace `english' with e.g. `spanish' to change the document language
\usepackage[english]{babel}

% Set page size and margins
% Replace `letterpaper' with `a4paper' for UK/EU standard size
\usepackage[letterpaper,top=2cm,bottom=2cm,left=3cm,right=3cm,marginparwidth=1.75cm]{geometry}

% Useful packages
\usepackage{amsmath}
\usepackage{graphicx}
\usepackage[colorlinks=true, allcolors=blue]{hyperref}


\title{Computer Networks - Homework 1}
\author{JJ McCauley}

\begin{document}
\maketitle

Chapter 1's Problems: 6,10,11,12,20,31

% Labeling sections with question number (adjusting counter)
\setcounter{section}{5}
\section{Available residential access technologies in Salisbury}

In Salisbury, there are various residential access technologies with various rates and speeds.

\subsection{Cable Internet: Xfinity}

Xfinity, a very popular provider in Salisbury, has numerous different plans with various downstream, upstream, and monthly prices.
\begin{enumerate}
    \item "Connect" Plan \begin{enumerate}
        \item Downstream Rate: 300 Mbps
        \item Upstream Rate: 5 Mbps
        \item Monthly Price: Around \$30
    \end{enumerate}
    \item "Gigabit" Plan \begin{enumerate}
        \item Downstream Rate: 1000 Mbps
        \item Upstream Rate: 15 Mbps
        \item Monthly Price: Around \$65-\$80 
    \end{enumerate}
    \item "Gigabit Extra" \begin{enumerate}
        \item Downstream Rate: 1200 Mbps
        \item Upstream Rate: 35 Mbps
        \item Monthly Price: Around \$75-\$105
    \end{enumerate}
    \item "Gigabit x2" \begin{enumerate}
        \item Downstream Rate: 2000 Mbps
        \item Upstream Rate: 300 Mbps
        \item Monthly Price: Around \$95
    \end{enumerate}
\end{enumerate}

\subsection{5G Home Internet: T-Mobile}
T-Mobile offers various plans with different metrics, per \href{https://www.t-mobile.com/home-internet/policies/internet-service/network-speed-performance-metricsper}{T-Mobile's Metrics}.
\begin{enumerate}
    \item "Rely" Plan \begin{enumerate}
        \item Downstream Rate: 87-318 Mbps
        \item Upstream Rate: 14-56 Mbps
        \item Monthly Price: Around \$50
    \end{enumerate}
    \item "Amplified" Plan  \begin{enumerate}
        \item Downstream Rate: 133-415 Mbps
        \item Upstream Rate: 12-55 Mbps
        \item Monthly Price: Around \$60
    \end{enumerate}
    \item "All-in" Plan \begin{enumerate}
        \item Downstream Rate: 133-415 Mbps
        \item Upstream Rate:12-55 Mbps
        \item Monthly Price: Around \$70
    \end{enumerate}
\end{enumerate}
In contrast to the wide-spread availability of Xfinity's cable internet, 5G home internet has significantly less availability, which is important to note.

\subsection{Fiber Optic: Glofiber}
In addition to other providers, Glofiber has various plans with different speeds and prices.
\begin{enumerate}
    \item \$70/month Plan \begin{enumerate}
        \item Downstream Rate: 600 Mbps
        \item Upstream Rate: ?
    \end{enumerate}
    \item \$85/month Plan \begin{enumerate}
        \item Downstream Rate: 1200 Mbps
        \item Upstream Rate: ?
    \end{enumerate}
    \item \$140/month Plan \begin{enumerate}
        \item Downstream Rate: 2400 Mbps
        \item Upstream Rate: ?
    \end{enumerate}
    \item \$290/month Plan \begin{enumerate}
        \item Downstream Rate: 5000 Mbps
        \item Upstream Rate: ?
    \end{enumerate}
\end{enumerate}

\subsection{Other Providers}
In addition to the more well-known providers, there are various other providers in Salisbury.
\begin{itemize}
    \item Fixed Wireless Internet: Bloosurf \begin{enumerate}
        \item Downstream Rate: Up to 100 Mbps
        \item Upstream Rate: ?
        \item Monthly Price: Not specified
    \end{enumerate}
    \item Satellite Internet: HughesNet \begin{enumerate}
        \item Downstream Rate: Up to 100 Mbps
        \item Upstream Rate: ?
        \item Monthly Price: Starts at \$49.99
    \end{enumerate}
    \item Satellite Internet: Starlink \begin{itemize}
        \item Downstrate Rate: Up to 220 Mbps
        \item Upstream Rate: ?
        \item Monthly Price: Starting at \$120
    \end{itemize}
\end{itemize}

\setcounter{section}{9}
\section{Comparing popular wireless internet access technologies}
Today, the most popular wireless internet access technologies are Wi-Fi (WLAN), Celluar Networks, Fixed Wireless Access (FWA), and Satellite Internet.

\begin{table}[H]
    \centering
    \begin{tabular}{|p{3.5cm}|p{3.5cm}|p{3.5cm}|p{3.5cm}|}
        \hline
        \textbf{Wi-Fi} & \textbf{Cellular Networks} & \textbf{Fixed Wireless} & \textbf{Satellite} \\
        \hline 
        A short-range wireless technology that allows multiple devices to connect to the internet through a router connected to a broadband service.
        & Cellular Internet provides mobile broadband through cell towers, allowing connection anywhere with coverage
        & Fixed wireless delivers broadband to homes and businesses using radio signals from nearby towers.
        & Satellite internet transmits data between a satellite dish and orbiting satellites \\
        \hline
        Very fast speeds, up to 9.6 Gbps
        & Moderately fast speeds, around 10-3000 Mbps
        & Slower speeds, typically around 25-1000 Mbps
        & Slow speeds, around 25-250 Mbps \\
        \hline
        Availability Limited to router range
        & Wide/Expanding coverage
        & Limited Availability (requires nearby tower)
        & Available almost anywhere \\
        \hline
        Best for home, office, and public spaces
        & Best for mobile internet, rural areas, and home internet (5G)
        & Best for rural homes and businesses
        & Best for remote and very rural areas where access is limited \\
        \hline
    \end{tabular}
    \caption{Wireless Internet Access Technologies Comparison}
    \label{tab:my_label}
\end{table}
In summary: \begin{itemize}
    \item \textbf{Wi-Fi} is best used for local, high-speed wireless internet
    \item \textbf{4G/5G Celluar Networks} are best for mobility
    \item \textbf{Fixed Wireless} offers a strong broadband alternative in areas without fiber or cable capabilities
    \item \textbf{Satellite} is a last-resort option for those who don't have access to any other networks
\end{itemize}
Each wireless internet access technology has various different use cases, however Wi-Fi is commonly the most popular and widespread used in non-remote areas for fast internet connectivity.

\section{Determining end-to-end delay for packet of Length L}
Assuming that there is exactly one packet switch between the sending and receiving host and that the switch uses a \textbf{store-and-forward approach}, the total end-to-end delay for a packet of length L would be as follows:
\[
d_{\text{total}} = \frac{L}{R_1} + \frac{L}{R_2}
\]
where \( R_1 \) and \( R_2 \) represents the transmission rates between the host and the switch, and the switch to the host, respectively.   
Since the store-and-forward approach delay happens twice in a two-link network (the switch must fully receive the packet before forwarding it), the total delay formula are the two delay formulas for each switch added together.

\section{Advantages of Circuit Switching Techniques}
\subsection{Circuit-Switching vs. Packet-Switching Networks}
In circuit switching, a dedicated communication path is established prior to transmission, remaining open for the entire session and reserving bandwidth for the duration. In contrast, Packet-Switching does not set up a dedicated path, with each packet potentially taking different routes and only using resources as needed. Although Packet-Switching consumes less resources, Circuit-Switching has significantly less latency, no packet loss, and ensures consistent performance (not subject to performance impacts from congestion).
\subsection{TDM vs FDM in Circuit-Switched Networks}
Both TDM (Time-Division Multiplexing) and FDM (Frequency-Division Multiplexing) are used in circuit-switching networks, aiming to allow multiple users to share a network. FDM allows for users to transmit data in alternating turns, while FDM assigns each user a separate channel. Unlike FDM, TDM is advantageous in that it dynamically allocates bandwidth and TDM avoids interference. 
\setcounter{section}{19}
\section{Describing packet process}
If end system A wants to send a large file to end system B, end system A will firstly split the file into smaller chunks, then add headers to each split with important information, then send each packet off individually to reach end system B. When each of these packets arrives to a router, the router will look at the destination IP address to determine the best path to forward the packet. Packet switching is analogous to driving from one city to another in the sense that there is no dedicated road to drive on (no fixed path for each packet), the driver will ask for directions and reroute at each step (router determines where to send each packet based on network conditions), dynamic paths are based on conditions (such as a congested link), and different packets may take different routes to arrive at a destination, although they may arrive in a different order.

\setcounter{section}{30}
\section{}

\end{document}